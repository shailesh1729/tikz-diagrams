\documentclass[../preview.tex]{subfiles}
\begin{document} 
\begin{figure}
\centering
\begin{tikzpicture}[
    >=latex',
    font={\footnotesize},
    every node/.style={
        draw,
        minimum width=2cm,
        minimum height=0.5cm,
        align=left,
        scale=0.8
    },
    comment/.style={
        draw=none, 
        rectangle,
        text=black!80!
    }, 
    node distance=3mm,
    scale=0.8
]
\node (terminal1) [terminal] {START};
\node (predproc1) [below=of terminal1, predproc]
{Data set $X$};
\node (process1) [below=of predproc1, process]{
    Initialize\\dictionary $\mathcal{D}$
};
\node (process2) [below=of process1, process]{
    Initialize sparse\\approximations $\mathcal{A}$
};
\node (process2-comment)[right=of process2, comment]{
    By solving $X \approx \mathcal{D} \mathcal{A}$\\ using some
    suitable\\ greedy sparse\\ recovery algorithm
};
\node (process3) [below=of process2, process]{
    Update dictionary $\mathcal{D}$
};
\node (process4) [below=of process3, process]{
    Update sparse\\approximations $\mathcal{A}$
};
\node (decision1)[below=of process4, decision] {
    Converged?
};
\node (decision1-comment)[right=of decision1, comment]{
    Check the reduction\\in approximation error
};
\node (terminal2)[below=of decision1, terminal]{
    END
};

\draw[->] (terminal1.south) -- (predproc1);
\draw[->] (predproc1) -- (process1);
\draw[->] (process1) -- (process2);
\draw[->] (process2) -- (process3);
\draw[->] (process3) -- (process4);
\draw[->] (process4) -- (decision1);
\draw[->] (decision1.west) -- +(-1cm, 0cm) |- node[draw=none, near start, xshift=-5mm]{No} (process3.west);
\draw[->] (decision1) -- node[draw=none, xshift=5mm]{Yes} (terminal2);


\end{tikzpicture}
\end{figure}
\end{document}

\documentclass[../preview.tex]{subfiles}
\begin{document} 
\begin{figure}
\begin{tikzpicture}[
cell/.style={
  draw=blue,
  shape=rectangle,
  minimum size=5mm,
  fill=red!10
}
]
% image
\draw (0, 0) rectangle (3,3);
% image is 30mm by 30mm 6x6 cells. each cell is 5mm x 5mm
\node[cell](na)  at ($(4*5mm,1*5mm) + (2.5mm,2.5mm)$){};
\node[cell](nb)  at ($(2*5mm,2*5mm) + (2.5mm,2.5mm)$){};
\node[cell](nc)  at ($(4*5mm,4*5mm) + (2.5mm,2.5mm)$){};
\node at (15mm, 5cm) {
\begin{tabular}{c}
Input Image\\
or input feature map
\end{tabular}
};
% CNN layer
\begin{scope}[xshift=5cm]
\foreach \r in {0,...,3}{
  \fill[opacity=0.2]  ($(0, 0) + (\r*6mm, \r*6mm)$) rectangle +(2cm, 2cm);
}
\node at (2cm, 5cm) {Output Feature Maps};

\coordinate(ca) at (18mm, 5mm);
\coordinate(cb) at (16mm, 15mm);
\coordinate(cc) at (35mm, 35mm);
\draw (na.north) -- (ca)  (na.south) -- (ca);
\draw (nb.north) -- (cb)  (nb.south) -- (cb);
\draw (nb.north) -- (cb)  (nb.south) -- (cb);
\draw (nc.north) -- (cc)  (nc.south) -- (cc);
\end{scope}


\end{tikzpicture}
\centering
\end{figure}
\end{document}

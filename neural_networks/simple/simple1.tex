\documentclass[../preview.tex]{subfiles}
\begin{document} 
\begin{figure}
\centering
\begin{tikzpicture}[
neuron/.style={
  circle,
  minimum size=10mm,
  draw=blue,
  outer sep=2pt,
  fill=red!20,
},
every edge/.style={
  -{Straight Barb[angle=60:2pt 3]},
  draw=green!50!black,
} 
]
\def\InputGap{15mm}
% layer 1 input
\begin{scope}[node distance=2mm]
\node[neuron] (a10) {};
\node[neuron, below=of a10] (a11) {};
\node[neuron, below=of a11] (a12) {};
\node[neuron, below=of a12] (a13) {};
\node[neuron, below=of a13] (a14) {};
\end{scope}
% layer 0
\coordinate[left=\InputGap of a10] (i0) {};
\coordinate[left=\InputGap of a11] (i1) {};
\coordinate[left=\InputGap of a12] (i2) {};
\coordinate[left=\InputGap of a13] (i3) {};
\coordinate[left=\InputGap of a14] (i4) {};
% layer 2 hidden
\begin{scope}[node distance=4mm]
\node[neuron, right=20mm of a12] (a21) {};
\node[neuron, above=of a21] (a20) {};
\node[neuron, below=of a21] (a22) {};
\end{scope}
% layer 3 output
\node[neuron, right=20mm of a21] (a30) {};
\coordinate[right=\InputGap of a30] (out) {};

% layer names
\node[above=1mm of a10] (t1)
{\begin{tabular}{c}Input\\layer\end{tabular}};

\node[](t2) at (t1-|a20)
{\begin{tabular}{c}Hidden\\layer\end{tabular}};

\node[](t3) at (t1-|a30)
{\begin{tabular}{c}Output\\layer\end{tabular}};

% edges
\draw 
  (i0) edge node[above]{input 0} (a10)
  (i1) edge node[above]{input 1} (a11)
  (i2) edge node[above]{input 2} (a12)
  (i3) edge node[above]{input 3} (a13)
  (i4) edge node[above]{input 4} (a14)

  (a10) edge (a20) edge (a21) edge (a22)
  (a11) edge (a20) edge (a21) edge (a22)
  (a12) edge (a20) edge (a21) edge (a22)
  (a13) edge (a20) edge (a21) edge (a22)
  (a14) edge (a20) edge (a21) edge (a22)

  (a20) edge (a30)
  (a21) edge (a30)
  (a22) edge (a30)

  (a30) edge node[above]{output} (out)

  ;
\end{tikzpicture}
\end{figure}
\end{document}
